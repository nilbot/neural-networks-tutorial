\documentclass{scrartcl}
\usepackage{tikz}
\usepackage{forest}
\usepackage{amsmath}
\usepackage{amsfonts}
\usepackage{enumerate}
\usepackage{paralist}
\newcommand\numberthis{\addtocounter{equation}{1}\tag{\theequation}}
\usepackage{amsthm}
\usepackage{titlesec}
\usepackage{listings}

%% Golang definition for listings
%% http://github.io/julienc91/lstlistings-golang
%%
\RequirePackage{listings}

\lstdefinelanguage{Golang}%
  {morekeywords=[1]{package,import,func,type,struct,return,defer,panic,%
     recover,select,var,const,iota,},%
   morekeywords=[2]{string,uint,uint8,uint16,uint32,uint64,int,int8,int16,%
     int32,int64,bool,float32,float64,complex64,complex128,byte,rune,uintptr,%
     error,interface},%
   morekeywords=[3]{map,slice,make,new,nil,len,cap,copy,close,true,false,%
     delete,append,real,imag,complex,chan,},%
   morekeywords=[4]{for,break,continue,range,goto,switch,case,fallthrough,if,%
     else,default,},%
   morekeywords=[5]{Println,Printf,Error,},%
   sensitive=true,%
   morecomment=[l]{//},%
   morecomment=[s]{/*}{*/},%
   morestring=[b]',%
   morestring=[b]",%
   morestring=[s]{`}{`},%
   }

\usepackage{wrapfig}
\usetikzlibrary{automata,arrows,positioning}
\usepackage[toc,page]{appendix}
\title{Summary: Finding Structure in Time \\
 \large Assignment 2 COMP30230}
\author{Ersi Ni\\
\large 15204230}
\usepackage[parfill]{parskip}

\begin{document}
\maketitle


Jeff Elman addresses the problem of "Finding Structure in Time", proposing a way of representing temporal features implicitly rather than explicitly. Representing time is a general challenging problem, as many researchers from different domains have come to agree on the notion based on different reasons. 

Elman pointed out that representing time explicitly has few short-comings. Time as extra dimension is not a biologically inspired model and we have difficulties in describing duration, lapse and other properties of time. 

To represent time implicitly a neural nets can utilise memory, such that the memory components help the nets to remember the context and exhibit properties put forward by time as temporal effect on processing itself. Through a reinterpretation of the original XOR problem statement in a temporal fashion, the context augmented neural nets shows evidence that it can learn the temporal structure of XOR sequence. 

This raised question whether the size of memory affects how complex of a problem can the neural nets learn. Through another experiment with more complex input and variable duration Elman conjectured that the structured dependencies are actually easier to learn, in other words, larger and more complex memory construct may not be a necessity. 





\end{document}